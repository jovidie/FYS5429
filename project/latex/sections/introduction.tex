%================================================================
\section{Introduction}\label{sec:introduction}
%================================================================
% Neural networks have found many areas of use, and its use only increases as computational resources increase. Several companies have set up farms of servers around the world to distribute the use of energy. Recently, Google announced that they are building a new data center in Skien in Norway. 

\textbf{Draft introduction}
In neuroscience, machine learning has become a useful tool, as it can ie. reduce dimensionality in data \cite{Badrulhisham:2024:ml_and_ai_in_neuroscience}. It allows us to investigate several hypotheses, before testing the the most promising ones in animal models. This process can speed up the time of testing, while reducing the number of animal lives sacrificed. 

One machine learning method is artificial neural network. It was inspired by the synapses in the brain, and has been found useful in neuroscience. 
% Add more on machine learning
One interesting circuit to investigate, is the entorhinal-hippocampal circuit, which is thought to be vital in navigation \cite{okeefe:1978:hippocampus, hafting:2005:microstructure}. Using biological plausible conditions, the neural network can learn how to path integrate \cite{banino:2018:vector_based}.

The aim of this project is to train a neural network to learn trajectories, using velocity data as input. Since the trajectories are time dependent, and the model take sequential data as input, I will implement the model using a recurrent neural network. In addition, I will use increase input and use both velocity and head direction, and compare the predicted trajectories.

First, I will present a theoretical background for the project in section \ref{sec:theory}. I will give a brief overview of the relevant neurobiological circuits, and artificial neural networks, before presenting the methods used in implementing the models in section \ref{sec:methods}. In section \ref{sec:results} I present the result, followed by a discussion in section \ref{sec:discussion}. Lastly, I conclude my findings in section \ref{sec:conclusion}, and include possible future research questions.

% The goal of basic research in neuroscience, is to understand how the brain works. As the brain is made up of three main parts - the cerebrum, cerebellum and brainstem. Many advances have been made, based on people suffering from injuries. Such as Phineas Gage who survived an iron rod through his skull, which damaged his frontal lobe. A change in Gage's behavior led scientists to understand the function of the frontal lobe in humans...

% Throughout history, several psychiatric treatments have been important in understanding the human brain. However, knowing the function of the main areas of the brain, is not enough in understanding the brain as a whole. Moving down to a molecular level, we still have a lot to learn. 

% Basic research in neuroscience aims to further understand of the human brain. In knowing the baseline we can more easily understand the mechanisms affected in a diseased brain, what happens to the healthy brain when an individual gets Alzheimer disease or a stroke damages the brain tissue. 

% In the hunt for answers it is common to use model organisms, and with increasing complexity of the research question it is often necessary to use mammals in order to study the behavior in both wild type and variants. However, in order to reach a point in the research where an animal experiment is both possible and necessary to further the understanding, we have to have a plausible hypothesis to test. 

The human brain is an extraordinary computer. It processes huge amounts of data during one day, which allow us to interact with and react to our surroundings. % https://www.britannica.com/science/information-theory/Physiology 
One fascinating feature is the ability to navigate and store memories, and a region important in spatial navigation and memory is the medial temporal lobe, where the entorhinal-hippocampal circuit lies. Early research found that lesions in the hippocampal area impaired the rat's ability to navigate %(Morris, R., Garrud, P., Rawlins, J. et al. Place navigation impaired in rats with hippocampal lesions. Nature 297, 681–683 (1982). https://doi.org/10.1038/297681a0) 
and taxi drivers who suffered stroke, damaging the hippocampus, could no longer recall trajectories \cite{maguire:2000:navigation}.

Recent studies have found that information on self position is computed in upstream from the hippocampus, and that the hippocampus itself is important in memory formation.

It is common to use model animals in research related to the human brain, as the ethical aspect of invasive methods make it difficult to use human as a model. In neuroscience, machine learning has become a useful tool, as it allows us to investigate several hypotheses, before testing the the most promising ones in animal models. This process can speed up the time of testing, while reducing the number of animal lives sacrificed.

Episodic memory is a type of declarative memory, which includes the ability to recall previous experiences. These memories include the element of what, when and where, 