%================================================================
\section{Introduction}\label{sec:introduction}
% Motivate the reader and present overarching ideas, and background on the 
% subject of the project. Mention what I have done and present the structure 
% of the report, that is how it is organized.
%================================================================
When we experience something new, we form a memory of that event. This is an episodic memory which relies on the elements what, when, and where. This is a declarative memory and all of the elements are linked into one event, like your first time riding a bicycle. You can likely recall the color of your bike, what season it was, and the street you rode in. Especially if you fell off the bike and hit yourself, which could have resulted in an emotional link to the memory.

The hippocampus is important in forming new episodic memories. This was discovered when a patient known as H.M. had both hippocampi surgically removed to stop epileptic seizures. After the surgery H.M. was not able to form any new memories, however, he was able to form procedural memories\footnote{Procedural memories consist of skills and habits.} \cite{scoville:1957:loss_recent}. In studies done with rats, hippocampal lesions\footnote{Lesion information} resulted in poor performance in navigation tasks \cite{kaada:1961:maze, schlesiger:2013:hippocampal_activation_maze}. In addition, London taxi drivers lost the ability to navigate the city after suffering from stroke damaging the hippocampus \cite{maguire:2000:navigation}.

How is the hippocampus important in spatial memory? Positional information is seen in the firing of so-called place cells, from recordings done in rats. When the rat performs a task such as running on a track, recording neuron activity in the hippocampus result in place fields. These can be ordered according to the rats position when a specific neuron fired, which result in a sequence of active neurons depending on time. These neurons are thought to encode a spatial map of the environment \cite{okeefe:1978:hippocampus}. 

The hippocampus receives input from the association cortex, which is where sensory modalities are processed. This is also where the entorhinal cortex lies. The activity recorded in the medial entorhinal cortex (MEC) form similar pattern as seen in place cells. However, this pattern is repetitive and the size differs within the layers of the MEC (seem to increase from dorsal to ventral). These neurons are called grid cells as they fire in a grid-like pattern when the rat moves around in a closed environment \cite{hafting:2005:microstructure}.

\begin{figure}
    \centering
    \includegraphics{}
    \caption{Caption}
    \label{fig:enter-label}
\end{figure}

The use of machine learning methods in neuroscience, have made it possible to test hypotheses using biologically plausible conditions. Recent studies have applied such methods, in trying to understand the connection between place cells and grid cells in navigation \cite{banino:2018:vector_based}. 

The aim of this project is to train a neural network to learn trajectories, using velocity data as input. Since a trajectory consist of sequential data, I will implement a recurrent neural network. After training the model on velocity data, I will increase the number of input features to include e.g. head direction, and compare the predicted trajectories. As a crude model of the entorhinal cortex-hippocamus circuit, I will implement a model using two hidden layers, to compare with the single layer model.

First, I present the methods used in implementing the models in section \ref{sec:methods}. In section \ref{sec:results} I present my  results and a discussion on the analysis. Lastly, I conclude my findings in section \ref{sec:conclusion}, in addition to possible future research questions.

